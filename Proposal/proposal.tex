\documentclass[a4paper,12pt,twoside,swedish]{report}

%\graphicspath{{figures/}}


\usepackage{times}  
\usepackage{amsmath}
\usepackage{amssymb}
\usepackage{graphicx}
\usepackage{theorem}
\usepackage[utf8]{inputenc}
\usepackage{latexsym}
\usepackage{url}
\usepackage{babel}

\headsep          0.25in
\headheight       12pt
\setlength{\topmargin}{-\headsep}
\addtolength{\topmargin}{-\headheight}
\oddsidemargin      0in
\evensidemargin     0in
\setlength{\textwidth}{\paperwidth}
\addtolength{\textwidth}{-2.0in}
\setlength{\textheight}{\paperheight}
\addtolength{\textheight}{-2.0in}
\addtolength{\textheight}{-1\topskip}
\divide\textheight  by \baselineskip
\multiply\textheight  by \baselineskip
\addtolength{\textheight}{\topskip}
\setcounter{secnumdepth}{-2}

\begin{document}

\pagestyle{plain}

\setcounter{page}{1}

\section{Inledning}
Målet med projektet är att skapa en fluidsimulering och visualisera i realtid. Simulering av olika fluider kan göras genom att exempelvis reglera de olika parametrarna i Navier-Stokes ekvationer och ändra renderingsmetod. 

\section{Beskrivning av systemet}
För fluidsimuleringen används Navier-Stokes ekvationer:\\
\(\frac{du}{dt} = - (\textbf{u}\cdot{\Delta})\textbf{u} - \frac{1}{\rho}\Delta p + v \Delta^2 \textbf{u} + \textbf{F}\) \\
\(\Delta \cdot \textbf{u} = 0 \) \\

\section{Simulering av systemet}
En förstudie kommer utföras i Matlab för att få en övergripande förståelse för hur systemet fungerar. Sedan kommer simuleringen implementeras i C++ med OpenGL och GLSL för att köras på GPU. Fluidsimulering lämpar sig mycket väl för en GPU-implementation då grafikkortet kan parallellisera beräkningarna, vilket ger bättre  prestanda.
 
\section{Implementation av animering}
Visualiseringen kommer implementeras som en volymsrendering på GPU. Eftersom vi strävar efter att animera olika fluider kommer olika egenskaper att behöva visualiseras.

\section{Dokumentation}

\section{Arbetsplanering}
	

\end{document}