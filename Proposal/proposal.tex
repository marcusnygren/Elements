\documentclass[a4paper,12pt,twoside,swedish]{report}

%\graphicspath{{figures/}}


\usepackage{times}  
\usepackage{amsmath}
\usepackage{amssymb}
\usepackage{graphicx}
\usepackage{theorem}
\usepackage[utf8]{inputenc}
\usepackage{latexsym}
\usepackage{url}
\usepackage{babel}

\headsep          0.25in
\headheight       12pt
\setlength{\topmargin}{-\headsep}
\addtolength{\topmargin}{-\headheight}
\oddsidemargin      0in
\evensidemargin     0in
\setlength{\textwidth}{\paperwidth}
\addtolength{\textwidth}{-2.0in}
\setlength{\textheight}{\paperheight}
\addtolength{\textheight}{-2.0in}
\addtolength{\textheight}{-1\topskip}
\divide\textheight  by \baselineskip
\multiply\textheight  by \baselineskip
\addtolength{\textheight}{\topskip}
\setcounter{secnumdepth}{-2}

\begin{document}

\pagestyle{plain}

\setcounter{page}{1}

\section{Introduction}


\section{Description of the physical system}

\(\frac{du}{dt} = - (\textbf{u}\cdot{\Delta})\textbf{u} - \frac{1}{\rho}\Delta p + v \Delta^2 \textbf{u} + \textbf{F}\) \\
\(\Delta \cdot \textbf{u} = 0 \) \\

\section{Simulation of the system}

\section{Implementation of the animation}
For example; during smoke simulation the temperature must be calculated at each point in the volume so that the appearance of hot smoke rising and cold smoke falling to the ground can be correctly modeled. In the case of simulating fire, it is required to keep track of how long it has been since the fuel was ignited. This provides the means to model how the fuel becomes exhausted.


\section{Documentation}

\section{Plan of the work}
	

\end{document}