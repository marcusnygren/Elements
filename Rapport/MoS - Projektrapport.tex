\documentclass[a4paper,12pt,twoside,swedish]{report}

\usepackage{times}  
\usepackage{amsmath}
\usepackage{amssymb}
\usepackage{graphicx}
\usepackage{theorem}
\usepackage[utf8]{inputenc}
\usepackage{latexsym}
\usepackage{url}
\usepackage{babel}

\headsep          0.25in
\headheight       12pt
\setlength{\topmargin}{-\headsep}
\addtolength{\topmargin}{-\headheight}
\oddsidemargin      0in
\evensidemargin     0in
\setlength{\textwidth}{\paperwidth}
\addtolength{\textwidth}{-2.0in}
\setlength{\textheight}{\paperheight}
\addtolength{\textheight}{-2.0in}
\addtolength{\textheight}{-1\topskip}
\divide\textheight  by \baselineskip
\multiply\textheight  by \baselineskip
\addtolength{\textheight}{\topskip}
\setcounter{secnumdepth}{-2}

\begin{document}

\pagestyle{plain}

\setcounter{page}{1}

\chapter{Modellbygge och Simulering - Elements}

\section{Sammanfattning}
\textit{En kort sammanfattning om hela rapporten. Ca 100 - 150 tecken. Ska skrivas sist.}

\section{Innehållsförteckning}
\textit{Rapportens innehållsförteckning}

\section{Introduktion}
\textit{Bakgrund - Projektets bakgrund, inspiration, research osv. Vad behövde vi veta för att skapa detta projekt?
Syfte - Projektets ursprungssyfte. Varför ville vi göra det vi gjorde?}

Elements är gränssnitt för att visa olika partikelsystem i en mjukvara och låta användaren experimentera med Navier-stokes ekvationer. Syftet är att användaren ska kunna se sambandet och förstå kopplingen mellan dessa fysikaliska ekvationer med en realtidsrenderad fluidsimulering på skärmen. Simuleringarna kommer i form av: Rök, eld och vatten.

\section{Introduktion}
\textit{Vad har vi använt för metoder? Formler? Beräkningar? Programmering? Hela workflowen i kort sammanfattning: Research - Matlab - C++ - OpenGL - QT. Vilka delar skedde parallellt?}

\subsection{Research}
\textit{Vad gjorde vi för research innan vi började hacka. Ekvationer? Förenklingar?}

\subsection{Matlab}
\textit{Allt vi gjort i Matlab och vad vi kom fram till med hjälp av detta. Varför använde vi Matlab? Fanns det andra metoder att gå tillväga? Hur fick vi över ekvationerna till Matlab?}

\subsection{C++}
\textit{Varför vi valde just C++? Hur gick vi tillväga med C++? Hur gick vi över ekvationerna till C++?}

\subsection{OpenGL}
\textit{Varför valde vi OpenGL? Hur fick vi fram renderingen på skärmen?}

\subsection{QT}
\textit{Varför valde vi QT? Hur kommunicerar QT med resten av projektet? C++ - QT? GUI?}

\section{Diskussion}
\textit{Vilka saker gick snett? Vilka saker gick strålande? Varför?}

\section{Resultat}
\textit{Projektets resultat. Om man följer hela workflowen i metod-delen vart hamnar man då? Vad kom vi fram till?}

\section{Referenser}
\textit{En lista på alla våra referenser i ordning där de används först i texten. Exempelvis: Papers på internet, böcker och andra externa informationskällor.}

\section{Bilaga A}
\textit{Här kan man lägga första bilagan om det behövs.}

\section{Bilaga B}
\textit{Här kan man lägga andra bilagan om det behövs. Fler bilagor blir bara nästa bokstav i alfabetet.}

\end{document}